%This research started as a \emph{side effect} of another investigation that we were carrying out.

To find out whether regression tests, created when a bug is fixed in a project, could help to analyze the \emph{origin} of that bug, we intended to run the test on past versions of the software.
The snapshot (commit) where the test fails would be the candidate for being the bug-introducing change~\cite{kim2006automatic}.
To do this, it was necessary to first build and then run the regression test on the previous snapshots of the project.
We found, however, that many of the snapshots were not buildable.

The dataset we used was the same as the one we were going to do in this study, Defects4J~\cite{Just:2014:DDE:2610384.2628055}.
Exploring other investigations that made use of this same dataset, we identified another study that had the same problems trying to build some snapshots~\cite{Just:2014:MVS:2635868.2635929}.
The authors decided to discard from their study those that did not build.
We discovered that in some cases it was possible to fix some of the build failures.
However, the diversity of errors motivated us to make a deeper analysis in the different projects, studying how buildable the snapshots were, and to create a taxonomy to understand and classify the most recurrent errors.

We are not the only ones interested in such an endeavor.
Both industrial software developers and software engineering researchers have shown interest in rebuilding past snapshots of projects~\cite{nikitin2017chainiac, RepBlds:2017:Online}, and do it in their day-to-day work.
Rebuilding the past snapshots of a software project is done with different objectives in mind, among others:

\begin{itemize}
	\item \textbf{{To search and find bugs}}: In line with our original intention, software developers often run previous snapshots of the system in order to localize bugs and understand how they originated~\cite{Zimmermann:2006:MVA:1137983.1138001}.
	\item \textbf{{Due to security reasons}}: Users usually trust the available binaries of a library, but a backdoor could be inserted~\cite{deCarnedeCarnavalet:2014:CIV:2664243.2664288}.
Rebuilding a library version from the original source allows to compare the binaries to verify that no new code has been introduced.
	\item \textbf{{To backport bug fixes}}: To apply a patch or an update to an older version of the software, it is necessary to be able to build that specific version~\cite{tian2017mining}.
	\item \textbf{{To reproduce the past state of a system}}: For research, it is often useful to obtain a functional binary/executable to verify the correct performance of the system~\cite{manacero2011using}.
	The use of the project history is also useful to predict future bugs~\cite{Zimmermann2008}.
\end{itemize}

%\grex{I don't understand why we put the `desirable features' here... this is something very specific that belongs to the method section or when we are presenting the case studies.}


% In the literature, build reproducibility is the ability to generate byte-to-byte identical binaries from the source code of the project, at any point of its history, no matter who builds the binary, when or in which machine~\cite{RepBldsDebian:2018:Online}.

Sulir and Porub\"an~\cite{Sulir:2016:QSJ:3001878.3001882} consider, in the context of projects created with compilable languages, the build process to be composed of the following steps:

\begin{enumerate}
	\item read the project build file
	\item download third party components defined in the build file
	\item execute the compiler to generate  binary files from source code, and
	\item package the program in a suitable format for deployment.
\end{enumerate}

For them, a specific project version is {\bf buildable} if these steps can be executed to generate a valid binary.
A related term is {\bf build reproducibility}, which is \emph{the ability to generate byte-to-byte identical binaries from the source code of a project version, no matter who builds the binary, when or in which machine}~\cite{RepBldsDebian:2018:Online}.
Reproducible builds create thus a verifiable path from human readable source code to the binary code used by computers, and are gaining relevance in recent times~\cite{cito2017empirical, maudoux2018correct}, in particular, but not limited, in relation to Blockchain technologies~\cite{deCarnedeCarnavalet:2014:CIV:2664243.2664288, perry2014reproducible}.
It should be noted that to generate a reproducible build from a specific version of a project, first that version should be buildable, and then additional conditions have to be met.
Build reproducibility may be pursued when security is a key characteristic, but there are many scenarios --as those pointed out above-- where having \emph{just} buildability should be \emph{enough}.

We define the {\bf buildability of a project} as \emph{the ability to build a project from its sources}.
Buildability is essential for any software project, as it has to be run.
In this work, we argue that buildability should not be limited to the current (i.e., latest) version of a software, but to all its history (or at least part of it) -- we refer to this ability as {\bf historic buildability}, and in particular to {\bf complete historic buildability} when \emph{all past versions of a software project can be built}.

Ideally, a version successfully built in the past should be built successfully nowadays.
However, the evolving nature of software has as a (non-intended) consequence that this is not always the case.
We have studied the buildability of the history of several projects checking if their snapshots are buildable or not.
We have focused on six well-known open source Java projects.
Five are taken from the Defects4J repository~\cite{Just:2014:DDE:2610384.2628055}, commonly used in Software Engineering research.
The other one is the Spring Framework project~\cite{Spring:2019:Online}, a big open source project commonly used in the industry for building web applications.
Specifically, this work tries to answer following research questions:

\begin{itemize}
	\item \textbf{RQ1}: Can we build all snapshots of a project? 
    This research question will shed some light into how \emph{historic buildable} software projects are. We consider a build as successful if a snapshot is buildable, and a fail if not.
	\item \textbf{RQ2}:  What are the most common problems that cause snapshot build fail?
    In addition to determine what problems prevent snapshots to be buildable, we propose a classification of the problems into different categories.
\end{itemize}

The contributions of this paper are following:
    
\begin{enumerate}
	\item To the best of our knowledge, this is the first work to address the buildability of all the snapshots of a project.
	\item We propose simple and generic tools to check the buildability of a project's history, and analyze the problems that make a snapshot build fail.
	\item We propose a taxonomy of build failures with an empirical base.
	\item We provide the extension of a previous dataset~\cite{Just:2014:DDE:2610384.2628055} including the history of buildable snapshots and the logs in the case in which they were not buildable for the projects studied.
\end{enumerate}

The work is structured as follows:
Previous work is presented in Section~\ref{sec:related}.
Subject project, tools and methodology is presented in Section~\ref{sec:metodology}.
Section~\ref{sec:results} and Section~\ref{sec:taxonomy} answer the research questions.
In Section~\ref{sec:discussion} we discuss about the work carried in this work.
Finally, Section~\ref{sec:conclusions} concludes the work.
